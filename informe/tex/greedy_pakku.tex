\section{Greedy - Algoritmo de Pakku}

Se implementó el algoritmo propuesto por el maestro Pakku.

Para determinar el "grupo con menos habilidad hasta ahora", se consideró
la suma de habilidades del grupo, por corresponder el mínimo de esta
suma al mínimo del cuadrado de la suma.


\subsection{Código Python}

\lstinputlisting[language=Python]{../greedy_pakku.py}

\subsection{Análisis de Complejidad}

\subsubsection*{Complejidad Temporal}


El algoritmo ordena la lista de habilidades de tamaño $n$,
con costo $\mathcal{O}(n \log n)$.

Luego, para cada uno de los $n$ elementos,
selecciona el grupo con menor suma actual mediante una
búsqueda lineal entre los $k$ grupos, lo que cuesta $\mathcal{O}(k)$
por iteración.

El costo total de esta fase es $\mathcal{O}(nk)$.

Por lo tanto, la complejidad temporal total es
\[
T(n,k) = \mathcal{O}(n \log n + nk).
\]

Para situaciones las que $k \ll n$, el término dominante
resulta $n \log n$, mientras que para $k$ grandes predomina el término lineal en $kn$.




\subsubsection*{Complejidad Espacial}

El algoritmo usa las siguientes estructuras:

\[
\text{(i) la lista ordenada: } \mathcal{O}(n), \qquad
\text{(ii) el arreglo de sumas: } \mathcal{O}(k), \qquad
\text{(iii) las particiones: } \mathcal{O}(n).
\]

En consecuencia, el uso total de memoria es
\[
S(n,k) = \mathcal{O}(n + k).
\]




\subsection{Calidad de la Aproximación}

El objetivo del problema consiste en particionar los $n$ maestros en $k$ grupos
$S_1,\dots,S_k$ de forma tal que, si denotamos por $L_j$ a la suma de habilidades
del grupo $j$, se minimice la función

\[
f(S_1,\dots,S_k) \;=\; \sum_{j=1}^{k} L_j^2.
\]

Para una instancia $I$, denotamos por $OPT(I)$ al valor óptimo de esta función y
por $A(I)$ al valor obtenido por el algoritmo Greedy de Pakku.

\subsubsection*{Cota inferior para el óptimo}

Sea $S = \sum_{i=1}^n x_i$ la suma total de habilidades. Para cualquier
partición de los maestros en $k$ grupos, se cumple

\[
\sum_{j=1}^{k} L_j = S.
\]

Usando que la función $x^2$ es convexa, obtenemos la siguiente cota inferior válida para \emph{toda} partición:

\[
\sum_{j=1}^{k} L_j^2 \;\ge\; k \left(\frac{1}{k}\sum_{j=1}^{k} L_j\right)^2
= k \left(\frac{S}{k}\right)^2
= \frac{S^2}{k}.
\]

Por lo tanto, para toda instancia $I$ se cumple

\[
OPT(I) \;\ge\; \frac{S^2}{k}.
\]

\subsubsection*{Cota superior para el algoritmo Greedy}

Analicemos ahora el comportamiento del algoritmo de Pakku.

\begin{itemize}
  \item ordena las habilidades de mayor a menor,
  \item y asigna cada maestro al grupo con menor suma actual.
\end{itemize}

Sea $p$ la habilidad máxima (es decir, $p = \max_i x_i$). En el momento en que
se asigna el maestro de habilidad $p$, la suma de habilidades ya asignadas es
$S - p$, con lo cual la carga promedio de los grupos en ese instante es

\[
\frac{S - p}{k}.
\]

Como el algoritmo elige siempre el grupo con menor suma, la carga de ese grupo
antes de asignar $p$ es, a lo sumo,

\[
L_{\text{antes}} \;\le\; \frac{S - p}{k}.
\]

Luego de agregar $p$, la carga de ese grupo queda acotada por

\[
L_{\text{después}}
\;\le\;
\frac{S - p}{k} + p
= \frac{S}{k} + \left(1 - \frac{1}{k}\right)p.
\]

En consecuencia, si denotamos por $M_G$ a la carga máxima en la solución Greedy,
tenemos

\[
M_G \;\le\; \frac{S}{k} + \left(1 - \frac{1}{k}\right)p.
\]

Por otro lado, como la suma de las cargas es $S$, podemos acotar el valor de la
función objetivo de la solución Greedy como

\[
A(I) = \sum_{j=1}^{k} L_j^2
\;\le\;
M_G \sum_{j=1}^{k} L_j
= M_G \cdot S
\;\le\;
\left(\frac{S}{k} + \left(1 - \frac{1}{k}\right)p\right) S.
\]

\subsubsection*{Factor de aproximación}

Combinando la cota inferior del óptimo con la cota superior del algoritmo, se obtiene

\[
\frac{A(I)}{OPT(I)}
\;\le\;
\frac{\left(\frac{S}{k} + \left(1 - \frac{1}{k}\right)p\right) S}
     {\frac{S^2}{k}}
=
1 + (k-1)\frac{p}{S}.
\]

Como $p \le S$ (la habilidad máxima no puede exceder la suma total), se tiene

\[
\frac{p}{S} \le 1
\quad\Rightarrow\quad
\frac{A(I)}{OPT(I)} \;\le\; 1 + (k-1) \;=\; k.
\]

Es decir, el algoritmo de Pakku es un algoritmo de \emph{aproximación $k$-aproximado}:
para cualquier instancia $I$ con $k$ grupos, el valor de la solución Greedy satisface

\[
A(I) \;\le\; k \cdot OPT(I).
\]

Además, la cota más fina

\[
\frac{A(I)}{OPT(I)} \;\le\; 1 + (k-1)\frac{p}{S}
\]

muestra que, cuando ningún maestro domina fuertemente la suma total
(es decir, cuando el cociente $p/S$ es pequeño), el factor de aproximación
real es mucho más cercano a $1$ que a $k$. En las mediciones experimentales
presentadas más adelante se observa justamente que, para las instancias
consideradas, el algoritmo Greedy produce soluciones muy cercanas al óptimo.




\subsection{Mediciones}

\subsubsection*{Tiempos de Ejecución - Datasets Grandes}
Para contrastar la estimación teórica de la complejidad temporal con observaciones
experimentales, se generaron diversos conjuntos de datos aleatorios de tamaños variables.
En total se realizaron mediciones para $17*5=85$ conjuntos distintos de habilidades, cada
uno con entre 200 y 1000 elementos.:

\begin{itemize}
  \item Se generó un conjunto de datos con 85 sets de tamaño comprendido entre 200 y 1000 elementos (habilidades).
  \item Cada habilidad con un valor comprendido entre 10 y 1000.
  \item Se definieron 50 grupos ($k=50$)
  \item Para cada set de ese conjunto de datos, se midió el tiempo de procesamiento.
\end{itemize}

Finalmente, se graficaron los tiempos medidos junto con la curva de ajuste correspondiente a la complejidad
teórica estimada.

\begin{figure}[H]
    \centering
    \includegraphics[width=.8\textwidth]{img/ajuste_greedy_pakku_muchos.png}
\end{figure}

Se observa que, si bien hay elevada dispersión en los resultados, los valores medidos se ajustan bien
al estimado teórico $n.log(n)$.










\subsubsection*{Comparación Tiempos - Backtraking vs. Greedy-Pakku}


Para observar el rendimiento del algoritmo Greedy-Pakku, se realizaron
experimentos que comparan su tiempo de ejecución con respecto al algoritmo exacto basado en Backtracking.  

Se evaluaron instancias con distintos valores de maestros $n$ y grupos $k$,
registrando tanto los tiempos de ejecución.  

Dado que el problema es NP-completo, el algoritmo exacto presenta un crecimiento
exponencial en función de $n$. Por este motivo en las pruebas se limitó su uso sets de
datos con $\leq20$.  

Por otro lado, la heurística Greedy mantiene tiempos de ejecución polinomiales.

\begin{figure}[H]
    \centering
    \includegraphics[width=.8\textwidth]{img/comparacion_tiempos_pakku_vs_bt.png}
\end{figure}

Nota: si bien los tiempos para el algoritmo Greedy parecen constantes, eso es debido
al pequeño intervalo considerado y a la gran diferencia de escala con los tiempos de
Backtraking. Como se vio en gráfico anterior, la solución Greedy tiene un orden $n.log(n)$







\subsubsection*{Ratio de la Aproximación}

Para analizar cuán cercana es la solución entregada por Greedy-Pakku respecto del valor
óptimo obtenido mediante backtracking, se realizaron múltiples experimentos sobre instancias
de distinto tamaño $n$. 

Las figuras siguientes resumen estos resultados a través de tres métricas complementarias:

\begin{itemize}
  \item Ratio Greedy/Óptimo: indica cuán lejos está la solución heurística
  del valor óptimo. Un ratio igual a 1 implica solución perfecta. El primer panel muestra
  la distribución completa de estos ratios para cada $n$ mediante boxplots.
  \item Peor, media y mediana del ratio: el segundo panel sintetiza el desempeño del Greedy
  para cada tamaño de instancia, destacando su comportamiento típico (media y mediana) y su
  peor caso observado.
  \item Diferencia absoluta (Greedy - Óptimo): el tercer panel muestra cuánto se desvía
  en valores absolutos la solución heurística. Se grafican la media, la mediana y el rango completo (mínimo -máximo) para cada $n$n.
\end{itemize}

En conjunto, estas métricas permiten evaluar tanto la estabilidad del método heurístico como
su ajuste relativo y absoluto respecto de la solución óptima.

\begin{figure}[H]
    \centering
    \includegraphics[width=.8\textwidth]{img/ratio_pakku_bt_1.png}
\end{figure}

\begin{figure}[H]
    \centering
    \includegraphics[width=.8\textwidth]{img/ratio_pakku_bt_2.png}
\end{figure}

\begin{figure}[H]
    \centering
    \includegraphics[width=.8\textwidth]{img/ratio_pakku_bt_3.png}
\end{figure}

Los resultados indican que Greedy-Pakku ofrece una adecuada aproximación al óptimo: 

Los ratios Greedy/Óptimo se mantienen cercanos a 1, tanto en su distribución (panel 1)
como en sus valores medio, mediano y peor caso (panel 2).  

Además, las diferencias absolutas entre ambas soluciones permanecen acotadas en todos
los tamaños evaluados (panel 3).  

Esto indica que el método heurístico logra soluciones de buena calidad con un costo computacional significativamente menor.










