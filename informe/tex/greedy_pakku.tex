\section{Greedy - Algoritmo de Pakku}

\textcolor{red}{TODO: Agregar Intro}

Se implementó el algoritmo propuesto por el maestro Pakku.  
Para determinar el "grupo con menos habilidad hasta ahora", se consideró
la suma de habilidades del grupo, por corresponder el mínimo de esta
suma al mínimo del cuadrado de la suma.

\lstinputlisting[language=Python]{../greedy_pakku.py}




\subsection{Mediciones}

\textcolor{red}{TODO: Agregar Contexto}

\subsubsection*{Tiempos de Ejecución - Datasets Grandes}

\textcolor{red}{TODO: Agregar Contexto}

\begin{figure}[H]
    \centering
    \includegraphics[width=.8\textwidth]{img/ajuste_greedy_pakku_muchos.png}
\end{figure}

\textcolor{red}{TODO: Agregar Interpretación}










\subsubsection*{Comparación Tiempos - Backtraking vs. Greedy-Pakku}

\textcolor{red}{TODO: Agregar Intro}

\begin{figure}[H]
    \centering
    \includegraphics[width=.8\textwidth]{img/comparacion_tiempos_pakku_vs_bt.png}
\end{figure}

\textcolor{red}{TODO: Agregar Interpretación}






\subsubsection*{Ratio de la Aproximación}

\textcolor{red}{TODO: Agregar contexto}  

\textcolor{red}{TODO: EL GRÁFICO ES MUY REDUNDANTE (además de feo)}

\begin{figure}[H]
    \centering
    \includegraphics[width=.8\textwidth]{img/pakku_vs_bt.png}
\end{figure}

\textcolor{red}{TODO: Agregar Interpretación}








