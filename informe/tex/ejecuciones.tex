\section{Ejemplo de Ejecuciones}
En esta sección se presentan los tests realizados tanto para el algoritmo de backtracking como para las aproximaciones greedy. El objetivo es verificar que el cálculo del coeficiente mínimo y la distribución de maestros en grupos sea correcto y analizar cómo se comportan los algoritmos aproximados frente a la solución óptima.

\subsection{Algoritmo de Backtracking}

\subsubsection{Test 1: Distribución general}
\textbf{Archivo:} \texttt{test\_uno.txt}

\begin{verbatim}
3
A, 8
B, 7
C, 6
D, 1
E, 2
F, 3
G, 5
H, 4
I, 2
J, 3
K, 4
L, 1
\end{verbatim}

\textbf{Resultados esperados:}
\begin{itemize}
    \item \emph{Coeficiente mínimo:} 706
    \item \emph{Grupos:}
    \begin{itemize}
        \item Grupo 1: A, B, L
        \item Grupo 2: C, G, K
        \item Grupo 3: H, J, F, I, E, D
    \end{itemize}
\end{itemize}

\subsubsection{Test 2: Parejas perfectas}
\textbf{Archivo:} \texttt{test\_parejas\_perfectas.txt}

\begin{verbatim}
4
Rohan, 300
Kya, 300
Bumi II, 300
Varrick, 300
Zhu Li, 200
Zaheer, 200
Ming-Hua, 200
Ghazan, 200
\end{verbatim}

\textbf{Motivación:} Comprobar distribución óptima en grupos con habilidades parejas exactas.

\textbf{Resultados esperados:}
\begin{itemize}
    \item \emph{Coeficiente mínimo:} 1000000
    \item \emph{Grupos:}
    \begin{itemize}
        \item Grupo 1: Varrick, Ghazan
        \item Grupo 2: Bumi II, Ming-Hua
        \item Grupo 3: Kya, Zaheer
        \item Grupo 4: Rohan, Zhu Li
    \end{itemize}
\end{itemize}

\subsection{Algoritmo de Aproximación Greedy}

\subsubsection{Test 1: Escalonado}
\textbf{Archivo:} \texttt{test\_escalonado.txt}

\begin{verbatim}
10
Maestro A1, 1000
Maestro A2, 1000
...
Maestro J10, 100
\end{verbatim}

\textbf{Motivación:} Distribución con habilidades en bloques decrecientes. Permite analizar cómo el algoritmo greedy maneja grandes diferencias entre grupos y si logra balancear adecuadamente.

\textbf{Resultados esperados:}  
\begin{itemize}
    \item \emph{Coeficiente:} 302500000.0
    \item \emph{Grupos:}
    \begin{itemize}
        \item Grupo 1: Maestro A10, Maestro B10, Maestro C10, Maestro D10, Maestro E10, Maestro F10, Maestro G10, Maestro H10, Maestro I10, Maestro J10
        \item Grupo 2: Maestro A9, Maestro B9, Maestro C9, Maestro D9, Maestro E9, Maestro F9, Maestro G9, Maestro H9, Maestro I9, Maestro J9
        \item Grupo 3: Maestro A8, Maestro B8, Maestro C8, Maestro D8, Maestro E8, Maestro F8, Maestro G8, Maestro H8, Maestro I8, Maestro J8
        \item Grupo 4: Maestro A7, Maestro B7, Maestro C7, Maestro D7, Maestro E7, Maestro F7, Maestro G7, Maestro H7, Maestro I7, Maestro J7
        \item Grupo 5: Maestro A6, Maestro B6, Maestro C6, Maestro D6, Maestro E6, Maestro F6, Maestro G6, Maestro H6, Maestro I6, Maestro J6
        \item Grupo 6: Maestro A5, Maestro B5, Maestro C5, Maestro D5, Maestro E5, Maestro F5, Maestro G5, Maestro H5, Maestro I5, Maestro J5
        \item Grupo 7: Maestro A4, Maestro B4, Maestro C4, Maestro D4, Maestro E4, Maestro F4, Maestro G4, Maestro H4, Maestro I4, Maestro J4
        \item Grupo 8: Maestro A3, Maestro B3, Maestro C3, Maestro D3, Maestro E3, Maestro F3, Maestro G3, Maestro H3, Maestro I3, Maestro J3
        \item Grupo 9: Maestro A2, Maestro B2, Maestro C2, Maestro D2, Maestro E2, Maestro F2, Maestro G2, Maestro H2, Maestro I2, Maestro J2
        \item Grupo 10: Maestro A1, Maestro B1, Maestro C1, Maestro D1, Maestro E1, Maestro F1, Maestro G1, Maestro H1, Maestro I1, Maestro J1
    \end{itemize}
\end{itemize}
\subsubsection{Test 2: Balance perfecto}
\textbf{Archivo:} \texttt{test\_balance\_perfecto.txt}

\begin{verbatim}
5
Aang, 100
Katara, 100
Sokka, 100
Toph, 100
Zuko, 100
Iroh, 100
Azula, 100
Mai, 100
Ty Lee, 100
Suki, 100
Jet, 100
Haru, 100
Bumi, 100
Pakku, 100
Jeong Jeong, 100
\end{verbatim}

\textbf{Motivación:} Todos los maestros tienen la misma habilidad.

\textbf{Resultados esperados:}  
\begin{itemize}
    \item \emph{Coeficiente mínimo (Greedy FFD):} 450000
    \item \emph{Distribución de grupos:}
    \begin{itemize}
        \item Grupo 1: Aang, Katara, Sokka
        \item Grupo 2: Toph, Zuko, Iroh
        \item Grupo 3: Azula, Mai, Ty Lee
        \item Grupo 4: Suki, Jet, Haru
        \item Grupo 5: Bumi, Pakku, Jeong Jeong
    \end{itemize}
\end{itemize}
