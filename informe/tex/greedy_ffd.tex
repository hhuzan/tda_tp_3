\section{Greedy - FFD}

Se implementó ...........

\subsection{Código Python}

\lstinputlisting[language=Python]{../aproximacion.py}

\subsection{Análisis de Complejidad}

\subsubsection*{Complejidad Temporal}


\subsubsection*{Complejidad Espacial}



\subsection{Mediciones}

\subsubsection*{Comparación Tiempos - Backtraking vs. Greedy-Pakku}

\textcolor{red}{TODO: Agregar Intro}

\begin{figure}[H]
    \centering
    \includegraphics[width=.8\textwidth]{img/comparacion_tiempos_ffd_vs_bt.png}
\end{figure}

Nota: si bien los tiempos para el algoritmo Greedy parecen constantes, eso es debido
al pequeño intervalo considerado y a la gran diferencia de escala con los tiempos de
Backtraking. Como se vio en gráfico anterior, la solución Greedy tiene un orden $n.log(n)$

\textcolor{red}{TODO: Ampliar}






\subsubsection*{Ratio de la Aproximación}

\textcolor{red}{TODO: Agregar contexto}

\begin{figure}[H]
    \centering
    \includegraphics[width=.8\textwidth]{img/aprox_vs_bt.png}
\end{figure}

\textcolor{red}{TODO: EL GRÁFICO ES MUY REDUNDANTE (además de feo)}

\textcolor{red}{TODO: Agregar Interpretación}
