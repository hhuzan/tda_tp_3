\section{Mediciones}

\subsection{Backtracking}

\textcolor{red}{TODO: Agregar contexto}

\subsubsection*{Con o Sin Ordenamiento Previo}

\begin{figure}[htp]
  \centering
  \begin{subfigure}[b]{0.8\textwidth}
    \centering
    \includegraphics[width=\linewidth]{img/ajuste_backtracking_sin_ordenar.png}
    \caption{Sin Ordenar}
  \end{subfigure}

  \vspace{2em}

  \begin{subfigure}[b]{0.8\textwidth}
    \centering
    \includegraphics[width=\linewidth]{img/ajuste_backtracking_ordenado.png}
    \caption{Datos Ordenados}
  \end{subfigure}

  \caption{Comparación de Tiempos de Ejecución Con o Sin Ordenamiento Previo}
\end{figure}

Se observa que ordenando en forma descendente los datos antes del realizar
el backtracking, los tiempos de ejecución se reducen a aproximadamente un tercio.

\textcolor{red}{TODO: Ampliar}















\newpage

\subsection{Greedy (Algoritmo de Pakku)}

\textcolor{red}{TODO: Agregar Contexto}

\subsubsection*{Dataset Grande}

\begin{figure}[H]
    \centering
    \includegraphics[width=.8\textwidth]{img/ajuste_greedy_pakku_muchos.png}
\end{figure}

\textcolor{red}{TODO: Agregar Interpretación}









\newpage
\subsection{Greedy Pakku vs. Backtracking}

\textcolor{red}{TODO: Agregar contexto}

\subsubsection*{Tiempos de Ejecución}

\begin{figure}[H]
    \centering
    \includegraphics[width=.8\textwidth]{img/comparacion_tiempos_pakku_vs_bt.png}
\end{figure}

\begin{figure}[H]
    \centering
    \includegraphics[width=.8\textwidth]{img/comparacion_tiempos_pakku_grande_vs_bt.png}
\end{figure}

\textcolor{red}{TODO: Agregar Interpretación}


\subsubsection*{Ratio de la Aproximación}

\textcolor{red}{TODO: Agregar contexto}  

\textcolor{red}{TODO: EL GRÁFICO ES MUY REDUNDANTE}

\begin{figure}[H]
    \centering
    \includegraphics[width=.8\textwidth]{img/pakku_vs_bt.png}
\end{figure}

\textcolor{red}{TODO: Agregar Interpretación}








