\section{Conclusiones}

El presente estudio comparativo entre enfoques exactos y heurísticos para el Problema de la Tribu del Agua evidencia la complejidad intrínseca de los problemas NP-completos y la relevancia de seleccionar estrategias algorítmicas acordes a la magnitud y características de los datos. La implementación del algoritmo de \textit{Backtracking} demuestra, de manera rigurosa, la capacidad de obtener soluciones óptimas a través de la exploración exhaustiva del espacio de soluciones, destacándose la importancia de las técnicas de poda y acotamiento para reducir la explosión combinatoria.

Por otro lado, las aproximaciones greedy, ofrecen un compromiso entre eficiencia computacional y calidad de la solución. La comparación de los resultados obtenidos por ambos métodos evidencia que la calidad de la solución depende críticamente del ordenamiento inicial de los elementos y de la estrategia específica de asignación utilizada, lo que subraya la necesidad de un análisis detallado sobre la aplicabilidad y las limitaciones de cada heurística en distintos tipos de instancias.

Finalmente, los resultados experimentales y la comparación de las distintas estrategias permiten extraer conclusiones generales sobre el comportamiento de los algoritmos en contextos combinatorios complejos: mientras que los métodos exactos aseguran optimalidad a un costo computacional elevado, las heurísticas proveen soluciones de alta calidad en tiempos prácticos, aunque con variabilidad dependiente de la instancia. 