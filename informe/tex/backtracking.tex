\section{Backtracking}

\textcolor{red}{TODO: Completar Intro}

Podas Implementadas:
\begin{itemize}
    \item Poda por Cota Superior:  
    \verb|if sum_cuad_actual >= mejor_valor: Podar|
    \item Poda de simetría básica para evitar permutaciones de grupos vacíos. \verb|ya_uso_vacio = False ....|
\end{itemize}


Mejoras sobre el algoritmo backtracking básico:
\begin{itemize}
    \item Ordenamiento previo de habilidades en forma descendente;
    tiende a encontrar mejores soluciones temprano y permitir mayor poda.
    \item Actualización incremental de \verb|suma_cuad_actual|. Evita recalcular toda la suma en cada nodo.
\end{itemize}



\lstinputlisting[language=Python]{../backtracking.py}

\subsection{Mediciones}

\textcolor{red}{TODO: Agregar contexto}

\subsubsection*{Con o Sin Ordenamiento Previo}

Se muestra el efecto de realizar o no el Ordenamiento previo descendente de las habilidades. 

\begin{figure}[H]
    \centering
    \includegraphics[width=.8\textwidth]{img/comparacion_bt_ordenado_vs_no_ordenado.png}
\end{figure}

Se observa que ordenando los datos, en forma descendente, antes del realizar
el backtracking, los tiempos de ejecución se reducen a aproximadamente un octavo.

\textcolor{red}{TODO: Ampliar}