\section{Introducción}

Es el año 95 DG. La Nación del Fuego sigue su ataque, esta vez hacia la Tribu del Agua,
luego de una humillante derrota a manos del Reino de la Tierra, gracias a nuestra ayuda.
La tribu debe defenderse del ataque.

El maestro Pakku ha recomendado hacer lo siguiente: Separar a todos los Maestros Agua en 
$k$ grupos $(S_{1},S_{2},\cdots ,S_{k}$. Primero atacará el primer grupo. A medida que el primer
grupo se vaya cansando entrará el segundo grupo. Luego entrará el tercero, y de esta manera
se busca generar un ataque constante, que sumado a la ventaja del agua por sobre el fuego,
buscará lograr la victoria.

En función de esto, lo más conveniente es que los grupos estén parejos para que, justamente,
ese ataque se mantenga constante.

Conocemos la fuerza/maestría/habilidad de cada uno de los maestros agua, la cual podemos
cuantificar diciendo que para el maestro $i$ ese valor es $x_{i}$, y tenemos todos los
valores $x_{1},x_{2},\cdots ,x_{n}$ (todos valores positivos).

Para que los grupos estén parejos, lo que buscaremos es minimizar la adición de los cuadrados
de las sumas de las fuerzas de los grupos. Es decir:

\[
\min\sum_{i=1}^{k}\left( \sum_{x_{j}\in S_{i}}^{}x_{j} \right)^{2}
\]
 
El Maestro Pakku nos dice que esta es una tarea difícil, pero que con tiempo y paciencia podemos
obtener el resultado ideal.