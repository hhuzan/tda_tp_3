\section{El problema es NP}

Se demostrará que el problema pertenece a la clase de complejidad \textbf{NP},
presentando un \textit{certificador eficiente} capaz de verificar un
\textit{certificado} en tiempo polinomial.

\subsection*{Certificado}
Un vector $C = (S_{1}, \ldots, S_{k})$, donde cada $S_{j}$ es un subconjunto
de índices que indica a qué grupo pertenece cada elemento $x_{i}$.

\subsection*{Algoritmo del certificador}
El certificador debe verificar las siguientes condiciones:
\begin{itemize}
    \item Que la cantidad de subconjuntos $S_{j}$ sea exactamente $k$.
    \item Que cada elemento $x_{i}$ pertenezca a algún subconjunto $S_{j}$.
    \item Que ningún elemento $x_{i}$ aparezca en más de un subconjunto.
    \item Que la suma de los cuadrados de las sumas de los elementos de cada
    $S_{j}$ sea menor o igual a $B$.
\end{itemize}

\noindent
La implementación del certificador se encuentra en \texttt{certificador.py} \\

\lstinputlisting[language=Python]{../certificador.py}

\newpage
\subsubsection*{Justificación de la complejidad}

Sean:
\begin{itemize}
    \item $n$: la cantidad de elementos (maestros) a agrupar,
    \item $k$: la cantidad de grupos permitidos,
    \item $B$: el valor máximo permitido para la suma de los cuadrados.
\end{itemize}

El certificador realiza las siguientes verificaciones:
\begin{enumerate}
    \item Que la cantidad de grupos sea exactamente $k$.
    \item Que ningún elemento se repita en más de un grupo.
    \item Que todos los elementos estén asignados a algún grupo.
    \item Que la suma total de los cuadrados de las sumas de cada grupo se calcule correctamente.
    \item Que el resultado obtenido se compare con el valor límite $B$.
\end{enumerate}

El análisis de complejidad de cada paso es el siguiente:

\begin{center}
\begin{tabular}{|c|l|c|}
\hline
\textbf{Paso} & \textbf{Descripción} & \textbf{Complejidad} \\
\hline
1 & Comparar la cantidad de grupos con $k$ & $\mathcal{O}(1)$ \\
\hline
2 & Recorrer todos los elementos para verificar repeticiones & $\mathcal{O}(n)$ \\
\hline
3 & Verificar que no falte ningún elemento & $\mathcal{O}(1)$ \\
\hline
4 & Calcular la suma total de los cuadrados & $\mathcal{O}(n)$ \\
\hline
5 & Comparar el resultado con $B$ & $\mathcal{O}(1)$ \\
\hline
\end{tabular}
\end{center}

Por lo tanto, la complejidad total del certificador es:
\[
\mathcal{O}(1) + \mathcal{O}(n) + \mathcal{O}(1) + \mathcal{O}(n) + \mathcal{O}(1) = \mathcal{O}(n)
\]

\noindent
Es decir, el certificador se ejecuta en tiempo lineal con respecto a la cantidad de elementos $n$.
Dado que la verificación de una solución candidata puede realizarse en tiempo polinomial respecto del tamaño de la entrada,
el \textbf{Problema de la Tribu del Agua pertenece a la clase NP}.
